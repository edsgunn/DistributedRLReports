\documentclass[letterpaper,11pt]{report}
\usepackage[margin=1in,letterpaper]{geometry} % decreases margins
\usepackage{amssymb}
\usepackage{amsmath}
\usepackage{algorithm}
\usepackage{algpseudocode}
\usepackage{graphicx}
\usepackage{caption}
\usepackage{subcaption}
\usepackage{float}
\usepackage{setspace}
\usepackage{hyperref}

\newtheorem{theorem}{Theorem}

\doublespacing
\begin{document}

\title{Fourth Year Project Report \\ \textbf{Distributed Reinforcement Learning on The Edge}}
\author{Edward Gunn \\ Supervisor: Konstantinos Gatsis}
\maketitle 

\begin{abstract}
    Distributed systems are ubiquitous in the modern, automated world.
    These systems collect large amounts of data from which it is possible to learn how to act more efficiently.
    Agents within the system can communicate in order to learn more efficiently than any individual and to overcome computational bottlenecks associated with working on the edge.
    However, when distributed over a network the rate at which agents can communicate becomes a bottleneck for their rate of learning.
    It is therefore desirable to develop communication efficient algorithms for these use cases. A class of algorithms based on reinforcement learning show promising results in this domain, allowing for collaboration between agents to optimize a reward signal. We consider a number of existing algorithms of this nature, discussing their respective strenghts and weaknesses and conducting a numerical comparison between them. As well as this, we propose a new alroithm, evolution strategies with probabilistic communication (ESPC), to adjustably reduce the number of messages sent by evolution strategies (ES), one of the best performing and most versatile algorithms. We provide justification for this algorithm as well as a direct comparison between the ES and ESPC. We show that in some scenarios ESPC performs better than ES in terms of both quality of learning and amount of communication.
\end{abstract}

\tableofcontents
\chapter{Introduction}

\section{Motivation}
We are surrounded by distributed systems that make up a significant proportion of the infrastructure we have come to rely on. 
From social media to electrical power networks, it is essential we understand and improve their inner workings to allow them to cope with the ever-increasing demands of the modern world. 
Distributed systems provide great advantages over centralized ones. 
One advantage is scalability, by distributing a workload we are able to handle more data, more users, and more complex tasks.
Distributed systems are also more resiliant than centralized ones, they can continue to function when exposed to failures.
The also provide a boost to performance as they can process data faster and more efficiently through parallelization. 
However, as the size of these systems grow, they are, often bottlenecked by their communication bandwidth. To take full advantage of the scalability of distributed systems and reduce the cost associated with communication we must develop communication efficient algorithms that maximize the rate of learning while minimizing the number and size of the messages that are exchanged.

The field of reinforcement learning (RL), recently highlighted by some very high profile algorithms such as AlphaGo \cite{AlphaGo} and OpenAI Five \cite{OpenAIFive}, aims to find policies that maximize the reward achieved in an environment. The problem setting it provides is incredibly versatile, with applications including robotics \cite{TableTennis}, games \cite{AlphaStar}, control systems \cite{TriplePendulum}, natural language processing \cite{RLHF}, and computer vision \cite{Vision}. 

Inspired by this, distributed deinforcement learning (DRL) maintains the framework but distributes the learning process with the goal of speeding up training and allowing for the use of larger more complex models. However, as with all distributed systems, we may face a communication bottleneck leading to the desire to reduce how often agents communicate and how large the messages are when they do.
As well as this, in many scenarios in which the data itself is distributed across devices and sharing this data is prohibited, for example for privacy or legal reasons, we must find a way to train a model while adhereing to these constaints. This is the problem addressed in federated reinforcement learning (FRL).

In distributed systems the bandwith available for communication between nodes varies widely. It is therefore desirable to develop algorithms where we can trade communication for performance where necessary to suit the limitations imposed by the system.

\section{Current approaches}
\label{sec:CurrentApproaches}
To achieve good communication efficiency there exisits a number of different avenues for exploration, highlighed by recent work in this field. These include, thresholded communication strategies, gradient compression and black-box optimization.
In thresholded communication we only communicate information which is sufficiently interesting or will provide a sufficient update to the system. Inspired by event triggered control (ETC) EBCDQL \cite{EBCDQL} is a Q-learning based method where agents only communicate samples where the difference between the estimated value of a state-action pair and the actual observed value, known as the temporal difference (TD) error, exceeds the threshold defined by a temporally discounted sum of past TD errors. This means only samples which provide a significant update to the Q value will be used resulting in good communication efficiency. 
However, due to the use of a tabular Q function this solution is not feasible for problems with large or continuous state or action spaces.
Similar to this DAVIA \cite{DAVIA} uses a linearly approximated value function where the approximate gradient of the objective function with respect to the weights is only communicated each episode if the approximate update to the objective function exceeds a threshold which increases over time. 
LAPG \cite{LAPG} builds on distributed policy gradient methods, reducing the amount of communicaiton without reducing the quality of learing. It does this by only communicating approximate policy gradients if the squared norm of the increase in the difference between the last two samples of the policy gradient, known as inovation, from the last uploaded policy gradient exceeds a threshold known as the LAPG condition.

Gradient compression increses communcation efficiency by reducing the size of the messages sent by the system. This is demonstarted in \cite{TDCompression}, the TD-EF algorithm is proposed in which the system communicates a compressed version of the TD error and uses error feedback from the true TD error to update future communications. Under technical assumptions the authors give guarantees of convergence to the optimal policy.

Black-box optimization techniques benefit from a great avantage in terms of communiation efficiency as they only need to communicate a single scalar after each episode. This is because they consider the system as a pure input output process where the input is the policy parameters and the output is the discounted cumulative reward recieved. Using this they optimize directly in policy space.
The Evolution Strategies (ES) \cite{ES} algorithm has show very strong results on a range of challenging reinforcement learning benchmarks such as Atari \cite{Atari} and MuJoCo \cite{MuJoCo}. It approximates a policy gradient by taking samples of the total reward achieved in trajectories with parameters perturbed around the current parameters. This is highly parallelizable due to the minimal communication bandwith used and lack of computationally expensive calculations such as calculcating gradients using backpropagation. It is also very versatile, being effective on both discrete and continuous problems, however, in environments where all agents often receive the same reward it can perform poorly with little exploration. 
To address this NS-ES \cite{NS-ES} optimizes the novelty of the behaviours the agent exhibits, thus the system learns behaviours that are not like any seen in past iterations. Combining this with the goal of maximising reward NSR-ES weights the importance of optimizing reward and optimizing novelty leading to an agent which is less likely to get stuck in local optima. Finally NSRA-ES uses an adaptive scheme to weight novelty and reward meaning the agent achieves very impressive rewards while also avoiding local minima.
ES exhibits poor data efficiency as it discards large batches of reward samples immediately after use. In response to this IW-ES \cite{IWES} reuses old reward samples through importance weighting. This leads to an acceleration of learning, however, with large learning weights the algorithm can become unstable.



\section{Approach}
The problem this project considers is that of a number of distributed agents acting in paralell instances of the same environment where they can communicate with a central learner to jointly solve a reinforcement learning problem. 
This architecture is shown in figure \ref{fig:Architecture}

\begin{figure}
    \centering
    \includegraphics[width=0.6\columnwidth]{Figures/Architecture.png}
    \caption{The distributed reinforcement architecture used for this project. Agents interact with a single environment taking actions and recieving an observation of the next state and a reward back. Agents also interact with the central learner sending messages and recieving updated model parameters.}
    \label{fig:Architecture}
\end{figure}
Specifically we consider environments in which the task is episodic, meaning the agent will eventually reach a terminal state, however, when this occurs is unknown.
To evaluate how well a specific algorithm solves this problem we define metrics based on the quality of learning and the amount of communication exhibited during training.
We use these metrics to compare algorithms from section \ref{sec:CurrentApproaches} in a longitudinal study spanning multiple environments. We also perform a qualitative analysis of each of the algorithms highlighting thier particular strenghts and discussing what can be done to address their weaknesses.

As well as this analysis, in section \ref{sec:GeneralScheme}, we propose a new general adjustable probibalistic communications scheme for ES, one of the most promising algorithms, and discuss the reasoning behind it. We then demonstrate in section \ref{sec:NaiveScheme} how a number of assumptions can be made to make the scheme implementable in what we refer to as the naive scheme.

Finally we conduct experiments to compare this naive scheme with the original communication scheme and analyse the effect of varying its parameters. As well as this we look into how well the assumptions made by the naive scheme hold.

We show that the evolution strategies algorithm with probabilistic communication (ESPC) has strong performance compared to ES, often achieving higher rewards while only communicating a fraction of the amount. We theorize that this is due to the bias it exhibits towards higher rewards causing ESPC to explore promising areas of parameter space more quickly that ES.
As well as this we explore how varying ESPC's adjustable communication parameter affects its performance. We find that ESPC performs better with more communication. Finally we analyse how parameterizing the distribution we use to choose which samples are communicated affects the performance of ES. Specifically in the case of using a rolling mean and variance for this purpose we find that a shorter horizon provides a boost to performance.

\chapter{Background}

\section{Reinforcement learning}
\label{sec:RL}
Reinforcement learning (RL) is a subfield of machine learning that focuses on designing algorithms and models that allow an agent to learn through interacting with an environment. Agents are rewarded for achieving specific goals and punished for undesirable behaviours. RL has its roots in the field of psychology, where researchers studied the behaviour of animals and humans as they learned through positive and negative feedback. Due to its versatility RL is used in a number of fields such control theory, game theory and multiagent systems. It has been successfully applied to a wide range of problems, including game playing, robotics, and natural language processing. However, it remains a challenging and active area of research, particularly in complex and high-dimensional environments as well as environments in which there are multiple agents. 

In RL, an agent learns by receiving feedback in the form of rewards or punishments for each action taken in the environment. The agent interacts with the environment through a series of discrete time steps. At each time step, the agent observes the current state of the environment and chooses an action based on its current policy. The environment then transitions to a new state and provides the agent with a reward based on the chosen action and the new state. The agent updates its policy based on the reward signal, with the goal of maximizing expected cumulative discounted reward over time.

Formally, the problem of RL can be modelled as a Markov decision process (MDP), which can be described by the 5-tuple $\mathcal{M} = (\mathcal{S}, \mathcal{A}, \mathcal{P}, R,\gamma)$ where $\mathcal{S}$ is the state space, $\mathcal{A}$ is the action space, $\mathcal{P}$ is a set of action-dependent Markov transition kernels, $R: \mathcal{S} \rightarrow \mathbb{R}$ is a reward function and $\gamma \in [0,1]$ is the discount factor.
The actions an agent takes are determined by its policy $\pi: \mathcal{S} \rightarrow \mathcal{A}$. The policy can be alternatively defined as a distribution over actions $a \sim \pi(s)$. This can be advantageous as a non-deterministic policy can lead to greater exploration and can more easily avoid local optima. 
% However, this leads to complications in mathematical analysis as expectations have to be taken over actions as well as states.

We consider the episodic formulation of RL where the agent acts in the environment until it reaches some terminal state whereupon the episode ends.
In order to ensure episodes are of finite length we require a non-empty set of states $\Omega \subseteq \mathcal{S}$ such that a state $s \in \Omega$ is reached a final reward is issued and the episode terminates.
We also require that every policy $\pi$ has a non-zero probability of reaching $\Omega$ at some point in the trajectory starting from every state.


The value of a particular policy in a particular state is defined as the  expected cumulative discounted reward the agent will receive starting in that state
\begin{align*}
        V^\pi(s) &= \mathbb{E} [ \sum_{t=0}^\infty \gamma^t R(s_t)|s_0=s] \\
        &= R(s_0) + \gamma \mathbb{E} [V^\pi(s_1)],
\end{align*}
where $s_t$ is the state at time $t$ starting from $s_0=s$ and following policy $\pi$. The optimal value function is defined as that with the highest expected cumulative discounted reward 
\begin{equation*}
    V^*(s) = \max_\pi V^\pi(s),
\end{equation*}
which yields the optimal policy $\pi^*(s) = \arg \max_\pi V^\pi(s)$. 
We can find the optimal value function using value iteration by repeating the update
\begin{equation*}
    V_t(s) \leftarrow \max_{a \in \mathcal{A}} R(s) + \mathbb{E} V_{t-1}(s'),
\end{equation*}
for all states $s \in \mathcal{S}$ until convergence is achieved.
Value iteration is guaranteed to converge when the state and action spaces are finite, and reward values have a finite upper and lower bound.

It is common to come across environments where the state space is too large to use a tabular value functions as above. It will take too many iterations to have visited all the states to be able to have an accurate estimate of their value. It is thus necessary to use a function approximator so that all states can be represented. A common method for this is linear value approximation
\begin{equation*}
    \hat{V}_{\boldsymbol{\theta}}(s) = \sum_{k=1}^K \theta_k \phi_k(s) = \boldsymbol{\theta}^\top \boldsymbol{\phi}(s),
\end{equation*}
where $\boldsymbol{\theta}$ is the parameter vector and $\boldsymbol{\phi}(s)$ is a feature vector for state s. It is also common to use a neural network for this purpose.

To estimate the expected value of the state we need the transition probabilities between all states $p(s'|s,a), \ \forall s,s' \in \mathcal{S}, \ \forall a \in \mathcal{A}$. In practice, we often do not have an accurate model of the environment, so these are not known. We can instead use an action value function defined as 
\begin{equation*}
    Q_\pi(s,a) = \mathbb{E} [ \sum_{t=0}^\infty \gamma^t R(s_t)|s_0=s, a_1=a].
\end{equation*}
With this we can use Q-learning to approximate the optimal action value function with no direct model of the environment and from the action value function we can approximate the optimal policy. The Q-learning iteration given sample $(s,a,r,s')$ is 
\begin{equation*}
    Q^{new}(s,a) \leftarrow Q^{old}(s,a) + \alpha_t (R(s) + \max_{a'} Q(s',a')),
\end{equation*} 
where $\alpha$ is the learning rate. This is guaranteed to converge under the assumptions that the sum of the learning rates $\alpha_t$ diverge, the sum of their squares converge and each state-action pair is visited infinitely often \cite{QLearning}.

Alternatively to optimizing in value space we can optimize in policy space directly. One method for this is policy iteration where on each step we calculate the value function for the current policy then update the policy with 
\begin{equation*}
    \pi^{new}(s) \leftarrow \arg \min_{a \in \mathcal{A}} R(s) + \gamma V^{\pi^{old}}(s').
\end{equation*}
Similarly to with value functions we can approximate policies to simplify optimization. We can say $\pi(s) = \pi_\theta (s)$ where $\theta$ is a vector of parameters. This approximated policy is often a neural network meaning $\theta$ represents the weights and biases in the network. 
We define the objective function as the discounted expected cumulative reward from the initail state $s$ 
\begin{equation*}
    J^\theta(s) = \mathbb{E} [ \sum_{t=0}^\infty \gamma^t R(s_t)|s_0=s] = V^{\pi_\theta}(s)
\end{equation*}
Since the policy is now parameterized we can take its gradient with respect to some objective function and perform gradient ascent to optimize it. A well-known algorithm for this is REINFORCE \cite{REINFORCE} shown in algorithm \ref{alg:REINFORCE}.

\begin{algorithm}
    \caption{REINFORCE algorithm}\label{alg:REINFORCE}
    \begin{algorithmic}
            \State Initialize $\theta$ arbitrarily
            \For {each episode $\{s_0,a_1,r_1,\dots,s_{T-1}, a_T, r_T \} \sim \pi_\theta$}
                \For {$t=0,\dots,T-1$}
                    \State $\theta \leftarrow \theta + \alpha \nabla_\theta \log \pi_\theta(s_t,a_{t+1})v_t$ \Comment{$v_t=\sum_{k=t}^T \gamma^{k-t}r_k$}
                \EndFor
            \EndFor
    \end{algorithmic}
\end{algorithm}




\section{Federated learning}


Federated learning (FL) \cite{FederatedLearning} is a machine learning technique where training is performed in a decentralized manner, often on physically separate devices such as mobile phones, IoT devices or other edge devices. 
The critical issue federated learning attempts to address is that of training a robust model while preserving data privacy, data security, and data access rights.
A challenge that arises when faced with these constraints is the breakdown of the assumption that local data samples are identically distributed.
In FL we aim leverage the computing power and data generated by these distributed devices to build a shared model, while preserving the privacy and security of the data.

Federated learning can be broadly divided into four main phases, initialization, training, aggregation and testing. 
First during initialization phase the central server initializes a machine learning model, for example a deep neural network, and shares it with a set of devices. 
Next in the training phase, each device trains the model on its local training set, then evaluates the performance of the local model on the local evaluation set. 
The updated model parameters are sent back to the central server. 
The central server then aggregates these parameters and updates the shared model accordingly.
The updated model is sent back to the devices and this training and aggregation process is repeated until the desired level of accuracy is achieved on the local evaluation sets.
After this, in the testing phase the model is tested in a distributed manner on the local test sets. The performance of the model can then be aggregated to gain an overall performance metric for the model.

This process relies on several techniques to ensure the privacy and security of the data, such as encryption, differential privacy, and secure multi-party computation. This allows us to prevent the central server or any other party from accessing the raw data or the intermediate model parameters, while still allowing it to aggregate the updates and improve the shared model.

Federated learning has several advantages over traditional centralized machine learning approaches, including reduced data transfer costs, improved privacy and security, and the ability to learn from distributed and heterogeneous data sources. However, it also presents some challenges, such as communication and synchronization overhead, heterogeneity of devices, and model optimization across different data sources.

\section{Distributed reinforcement learning}

Distributed reinforcement learning (DRL) involves training an RL agent in a distributed manner across multiple machines, which may be physically separated, that work together to learn a single policy. DRL provides an advantage in large, complex environments or problems where single-agent RL might be too slow.
There exists two main paradigms of DRL: centralized training and decentralized execution (CTDE) and decentralized training and decentralized execution (DTDE).

In CTDE, a central learner is trained on the experience communicated from the agents. All agents share the same state and action space and collect trajectories through acting in an environment. The central learner then updates a shared policy based on the communications it receives from agents. 
% This architecture has been shown to be effective in multi-agent settings, where the actions of one agent affect the rewards of another.

In DTDE, each agent learns independently by acting in the environment. They maintain their own policy, updating it based on observations and rewards received from the environment. The agents communicate with each other directly to coordinate actions and improve performance. It is often more scalable and robust than CTDE, however, it can be less efficient in certain scenarios.

DRL is faced with issues when considering the communication and synchronization between agents. Centralized communication frameworks are often used, where a central server manages the communication and synchronization between the agents. An alternative to this is to use a decentralized communication framework, where agents communicate directly.

As well as this, DRL can take advantage of a number of existing RL algorithms, such as Q-learning, policy gradient methods, and actor-critic methods. These algorithms can be adapted for DRL by using mechanisms to handle distributed learning.

Federated reinforcement learning (FRL) is a subfield of DRL which adopts the philosophies underpinning federated learning. This includes the prohibition of sharing raw data, in this case taking the form of agent trajectories, between agents. They instead communicate data such as policy or value gradients.

\section{Exisiting algorithms}
\subsection{Distributed Q learning}
Distributed Q learning (DQL) adapts Q-learing to a distributed setting, increasing exploration and leading to an increase of the rate of convergence to the optimal policy. 
The agents communicate the trajectory collected after every episode.
The central learner determines the Q update by iterating through the agents and at each time step $t=1,\dots,n$ taking sample $u = (s_{t-1},a_t,r_t,s_t)$  and calculating $\Delta(u) = R(s_t) + \gamma \max_a Q(s_t,a) - Q(s_{t-1},a_t)$ which is the TD error.
Let the subsets $\mathcal{U}_s = \{u: s_{t-1}=s\}$.
The Q function is then updated by 
\begin{equation*}
    Q^{updated}(s,a) = Q^{old}(s,a) + \alpha \frac{1}{|\mathcal{U}_s|} \sum_{u \in \mathcal{U}_s} \Delta(u).
\end{equation*}

This method provides good convergence behaviour on simple environments, although it is not guaranteed to converge. It reasonably assigns credit for a reward to a given action even if the reward is delayed. However, due to the use of a tabular value function this method can perform poorly on environments with many state-actions as even after many iterations it will come across states it has seldom visited and thus does not have an accurate estimate of their value. As well as this the build up of a large table of Q values can fill the memory of the computer the algorithm is running on leading to a reduction in performance and eventually can cause it to crash.
Also, since the whole trajectory is communicated at each step the size of the messages sent can become very large for environments with long episode lengths or large state representations such as pixels on a screen meaning communication efficiency is poor.

Deep Q-learning is often used to address the performance issues where a neural network is used to approximate the Q function by estimating the value for each action in each state. This allows for use in larger, more complex environments. Data from a trajectory is used to approximate the gradient of the TD error with respect to the model parameters, allowing the use of gradient descent to minimize the error of the prediction. The model is often split into a prediction network, which approximates the value of the current state and a target network, which approximates the target from which we calculate the TD error.

\subsection{Event based communication DQL}
Event based communication distributed Q Learning (EBCDQL) \cite{EBCDQL} uses a communication scheme based on Event Triggered Control (ETC) \cite{ETC} techniques to reduce the communication of DQL while maintaining good learning performance.
The algorithm works similarly to vanilla DQL, however, agents only communicate the samples that have a TD error above the threshold defined as 
\begin{equation*}
    |\hat{\Delta}(u_i)| \geq \max(\rho L_i(t), \epsilon) 
\end{equation*}

where $L_i(t) = (1-\beta)L_i(t)+ \beta |\hat{\Delta}(u_i)|$ and $L_i(0)=0$.

This algorithm exhibits the same benefits and drawbacks as DQL, however, it has much better communication efficiency as it only sends samples to the central learner when they provide sufficient updates to the Q function.

\subsection{Distributed approximate value iteration algorithm}
In the distributed approximate value iteration algorithm (DAVIA) \cite{DAVIA}, agents communicate when a gradient step update provides a sufficient increase in the objective function, the threshold for which decreases over time. The objective function in this case is defined as the expected squared error between the updated value function and the current value function. 
\begin{equation*}
    J(\theta) = \mathbb{E}[V^{updated}(s) - \sum_{i=1}^n \theta_i \phi_i(s)]^2
\end{equation*} 
The algorithm uses a linearly approximated value function with an $\epsilon$-greedy policy, meaning it takes a random action with probability $\epsilon$ and the estimated best action otherwise.
The approximate gradient is calculated by
\begin{equation*}
    \hat{\nabla}J(\theta) = \frac{1}{T} \sum^T_{t=0} \phi(s_t)(\theta^T \phi(s_t) - r_t - \gamma V(s_{t+1}))
\end{equation*}
and the condition for communication is represented by 
\begin{equation*}
    J(\theta_k - \epsilon \hat{\nabla}J(\theta_k)) \leq \frac{\lambda}{\rho^{N-1-k}}.
\end{equation*}
In the central learner the gradient is updated by the mean of the transmitted gradients. This algorithm is guaranteed to converge under technical assumptions and shows good communication efficiency as only the gradients are communicated. The approximated value function also means it is able to handle more complex problems than DQL with larger state spaces. However, this does require pre-defined feature vectors for each state.

\subsection{Evolution Strategies}
Evolution Strategies (ES) \cite{ES} is a black-box optimization technique which approximates the policy gradient, or more specifically, the gradient of the expected cumulative discounted reward with respect to the parameters of a parameterized policy.
The agents each run an episode with parameters, perturbed normally about the current parameters $\theta$, before communicating the scalar cumulative discounted reward back to the central learner.
We can use the fact that
\begin{equation*}
        \nabla_\theta \mathbb{E}_{\epsilon \sim N(0,I)}[F(\theta+\sigma \epsilon)] = \frac{1}{\sigma}\mathbb{E}_{\epsilon \sim N(0,I)}[F(\theta+\sigma \epsilon) \epsilon]
\end{equation*}
where $F(.)$ is the cumulative discounted reward for a given value of the parameters and $\sigma$, chosen as a hyperparameter, is the standard deviation of the normally distributed perturbations $\epsilon$.
We calculate the approximate gradient by
\begin{equation}
        \hat{\nabla}_\theta = \frac{1}{n\sigma}\sum^n_{i=1} F_i \epsilon_i.
        \label{eq:grad}
\end{equation}
It is useful to note that since $\mathbb{E}_{\epsilon \sim N(0,I)}[F(\theta)\epsilon/\sigma] = 0$ we get

\begin{equation*}
    \mathbb{E}_{\epsilon \sim N(0,I)}[F(\theta+\sigma \epsilon) \epsilon/\sigma] = \mathbb{E}_{\epsilon \sim N(0,I)}[(F(\theta+\sigma \epsilon) - F(\theta))\epsilon/\sigma]
\end{equation*}
from which we can see that ES is computing the finite difference estimate of the gradient in a random direction. This suggests that this method will scale poorly with the dimension of the parameters $\theta$. However, in experiments this effect is not observed, in fact larger neural networks tend to perform better. The authors hypothesize this is because larger networks have fewer local minima \cite{LocalMinima}.

The main advantage of ES is its suitability for scalable parallelization. This is due to the minimal communication between agents and central learner as well as the absence of any backpropagation calculation. This means that ES can provide an order of magnitude speed up in training time. It also benefits from the fact that it only uses a parameterized policy so is suitable for any size of problem including those with continuous state and action spaces, however, by discretizing actions ES exhibits better exploration performance on some environments.

ES performs excellently in terms of the size of the messages sent, communicating only a single scalar, the cumulative discounted reward for the episode, per agent. However, since this communication occurs after every episode, the performance in terms of number of messages sent is poor. It also exhibits poor data efficiency, only updating its parameters once from large batches of trajectories.

% \subsection{Temporal difference learning with error feedback}

% Temporal difference learning with error feedback (TD-EF) \cite{TDCompression} allows the use of ...



\chapter{Comparative Analysis}

\section{Problem setup}

I consider a CTDE DRL problem with $N$ agents acting in parallel instances of the same environment.
The agents are able to episodically communicate with a central learner in order to jointly solve a Reinforcement Learning problem.
They collect time series data consisting of state-action trajectories and rewards.

Consider the episodic MDP $\mathcal{M}= (\mathcal{S}^N, \mathcal{A}^N, \mathcal{P}, R,\gamma)$ where the terms are defined as in section \ref{sec:RL}.
A trajectory following policy $\pi$ through the environment is denoted as a sequence $\zeta = (s_0, a_1, r_1, s_1, a_2, \dots)$.
At the end of episode $k$ in trial $l$ if some condition as a function of the current trajectory is satisfied i.e., $c^{k,l}_i=1$ where sample $c^{k,l}_i \leftarrow \mathcal{C}(\zeta^{k,l}_i), \ c^{k,l}_i \in \{0,1\}$ and $\mathcal{C}$ is some distribution representing a communication condition parameterized by the latest trajectory (this could be deterministic), then the agent $i$ communicates information $z^{k,l}_i=Z(\zeta^{1,l}_i,\zeta^{2,l}_i,\dots, \zeta^{k,l}_i)$, derived from the trajectories it has experienced, to the central learner such as select $(s_{t-1},a_t,r_t,s_t)$ tuples or value gradients.
The central learner then updates the policy based on the information received and communicates this back to the agents. 
The event loop can be seen in algorithm \ref{alg:EventLoop}.

\begin{algorithm}
    \caption{Distributed RL Event Loop}\label{alg:EventLoop}
    \begin{algorithmic}
            \State Initialize $L$ \Comment{The number of trials to be run}
            \State Initialize $N$ \Comment{The number of episodes to be run}
            \For {$l=1,\dots,L$}
            \State Initialize $\pi$
            \For {$k=1,\dots,N$} 
            \State central learner communicates $\pi$ to all agents
            \For {each agent $i = 1,\dots n$}
            \State run episode to collect sample trajectory $\zeta^{k,l}_i$ %\Comment{agents may adjust their policy during the episode}
            \State compute $z^{k,l}_i = Z(\zeta^{1,l}_i,\zeta^{2,l}_i,\dots, \zeta^{k,l}_i)$
            \State communicate $z^{k,l}_i$ if $c^{k,l}_i = 1$
            \EndFor
            \State update $\pi$ from received information
            \EndFor
            \EndFor
    \end{algorithmic}
\end{algorithm}

\section{Metrics}
\label{sec:Metrics}
To evaluate the quality of DRL algorithms I establish metrics which can be used to compare them directly. These metrics broadly cover how well the systems learn and how much they communicate. In terms of communication I exclusively focus on messages sent from the agents to the central learner as in this problem the communication from central learner to agents is fixed. To evaluate how well an agent learns we can average the discounted episodic reward achieved each episode across agents and trials then plot the average discounted episodic reward ($\mu^{\text{reward}}$) against episode number. The $\mu^{\text{reward}}$ as a function of episode number is 
\begin{equation*}
    \mu^{\text{reward}}(k) = \frac{1}{L} \sum_{l=1}^L \frac{1}{n} \sum_{i=1}^n \sum_{t} \gamma^t r^{k,l}_i(t)
\end{equation*}
where $r^{k,l}_i(t)$ is the reward in episode $k$ and trial $l$ for agent $i$ at time $t$ or equivalent the reward in trajectory $\zeta^{k,l}_i$ at time $t$.
We can say algorithm $a$ learns better than algorithm $b$ after $N$ training episodes if $\mu^{\text{reward}}_a(N) > \mu^{\text{reward}}_b(N)$.

As well as maximizing reward we would like to minimize communication. For this we define two metrics. The first is the number of messages a system has sent from agents to the central learner averaged over the number of agents ($\mu^{\text{frequency}}$). The $\mu^{\text{frequency}}$ at episode $k$  is
\begin{equation*}
    \mu^{\text{frequency}}(k) = \frac{1}{L} \sum_{l=1}^L \frac{1}{n} \sum_{i=1}^n c_i^{k,l}
\end{equation*} 
We say an algorithm $a$ communicates less frequently than algorithm $b$ on episode $N$ if $\mu^{\text{frequency}}_a(N) < \mu^{\text{frequency}}_b(N)$. The second is size in bytes of messages sent from agents to the central learner averaged over the number of agents ($\mu^{\text{size}}$). The $\mu^{\text{size}}$ at episode $k$ is 
\begin{equation*}
    \mu^{\text{size}}(k) = \frac{1}{L} \sum_{l=1}^L \frac{1}{n} \sum_{i=1}^n \text{bytes}(z_i^{k,l})
\end{equation*}  
We say algorithm $a$ communicates less intensely than algorithm $b$ on episode $N$ if $\mu^{\text{size}}_a(N) < \mu^{\text{size}}_b(N)$.

\section{Evaluation of existing algorithms}

\subsection{Computational setup}
\label{sec:AlgComp}
To conduct experiments it was necessary to implement each of the algorithms and to run them on various environments. To achieve this I built a framework in Python, specifying base classes for algorithms, environments and each of their respective components. The structure of the framework is shown in figure \ref{fig:CodeStructure} Each of the algorithms were implemented according to the template laid out by the framework allowing for any algorithm to be easily paired with any environment. ES uses a neural network policy which was implemented in PyTorch \cite{PyTorch} to enable acceleration of computation. Other algorithms mainly used NumPy \cite{NumPy} and SciPy \cite{SciPy} for their calculations. To pair algorithms with environments I implemented a Universe class which given a number of algorithms, environments, and parameters, appropriately pairs and evaluates them while recording details of the training via its logger. This means it was possible to run experiments in a large array of configurations using just a few lines of code in a Jupyter notebook. Once the experiments were complete the results were saved in a standardized data structure, shown in figure \ref{fig:DataStructure} from which I could analyse the performance of the algorithms. I used a number of Gymnasium (formerly OpenAI Gym \cite{Gym}) environments to train on as it has many standard benchmarks for RL algorithms pre-implemented. To make them compatible with my framework I wrote a wrapper that mapped the inputs and outputs to an appropriate form.
As a separate module I built a data analysis tool to extract the desired data from the results of an experiment and plot the results. This often involved extracting and combining data from multiple different files.

\begin{figure}[H]
    \centering
    \includegraphics[width=0.75\textwidth]{Figures/FrameworkStructure.png}
    \caption{The structure of the framework used to run the experiments}
    \label{fig:CodeStructure}
\end{figure}

\begin{figure}[H]
    \centering
    \includegraphics[width=0.75\textwidth]{Figures/DataStructure.png}
    \caption{The data structure used to store the results of experiments}
    \label{fig:DataStructure}
\end{figure}

\subsection{Experiments}
I evaluated the DQL, EBCDQL, DAVIA, and ES algorithms on the Simple Grid and Frozen Lake environments. They are both grid worlds where the goal is to move from a fixed starting state to a fixed terminal state and the possible actions are to move left, right, up, down, or stay in the same place. Simple Grid is a custom 5x5 grid world where the goal is to move from the top left state to the bottom right state, agents receive a reward of $-1$ for every step apart from at the terminal state where they receive a reward of $0$. Frozen lake (figure \ref{fig:FrozenLake}) is a 5x5 Grid environment from Gymnasium in which the goal is to move from the upper left to the lower right square upon which the agent receives a reward of $1$. A number of squares in the grid are holes in the lake, where if reached by the agent the episode ends. These environments were chosen as they were simple enough for all the algorithms to feasibly run on them while still providing a clear picture of each of their performance.

\begin{figure}
    \centering
    \includegraphics[width=0.6\textwidth]{Figures/frozen_lake.png}
    \caption{The Frozen Lake environment}
    \label{fig:FrozenLake}
\end{figure}

Each experiment used $n=5$ agents and a discount factor of $\gamma=0.9$. Agents were trained over 1000 episodes and this was repeated across 100 trials. I analyse the average discounted episodic reward ($\mu^{\text{reward}}$), the average number of messages ($\mu^{\text{frequency}}$), and the average size of messages ($\mu^{\text{size}}$). I plot the mean as well as the $10^{\text{th}}$ to $90^{\text{th}}$ percentile across trials for each algorithm.

The results on the Simple Grid environment are shown in figure \ref{fig:EvalSG}. It can be seen that DQL and DAVIA quickly converge to an optimal solution whereupon DAVIA rapidly reduces communication. EBCDQL reaches a suboptimal solution at which communication reduces significantly. This means the TD errors observed beyond this point are not large enough to justify communication even when improved performance is possible, resulting in a stagnation of learning. ES fails to learn an effective strategy for reaching the terminal, achieving near the minimum reward on every episode. This is likely due to a lack of exploration and low sensitivity to rewards when they are achieved. In this case the discount factor of $\gamma=0.9$ means that rewards towards the end of the trajectory have little impact resulting in a very small difference in rewards between agents, so exploration is only down to the randomly chosen perturbations and the approximation of the policy gradient is poor.

For ES and DQL the number of messages sent is, by design, one per agent per episode which in this setting is the worst achievable. For EBCDQL and DAVIA the number of messages sent decreases rapidly around episode 50 meaning fewer agents are receiving significant updates to their value functions each episode.

As DQL sends the whole trajectory after every episode, for long episode lengths the size of messages it sends is poor. This can be seen at the beginning of figure \ref{fig:EvalSGSizeMessages}. However, as the length of the episodes decrease, as faster routes to the terminal have been found, the size of the messages decreases dramatically. A similar behaviour is exhibited by EBCDQL, however, only significant subsections of the trajectory are sent, so the sizes of the messages are smaller. For DAVIA, since only the value gradient is communicated, the size of the messages sent are constant so the change in the average size of messages sent is only due to variations in the number of agents communicating. ES only communicates a single scalar per agent per episode, so the average size of messages is very small, but constant.

The results on the Frozen Lake environment are shown in figure \ref{fig:EvalFL}. Figure \ref{fig:EvalFLReward} shows the cumulative episodic reward rather than the episodic reward as the results are particularly noisy meaning the cumulative plot more clearly shows the learning of the agents. By looking at the gradient we can determine how well the algorithms learned. EBCDQL performed best followed by ES, however, after around episode 200 their respective plots become linear meaning no learning occurs from this point on. DQL and DAVIA both failed to reach the terminal state a significant number of times.

For EBCDQL and DAVIA in this environment communication was minimal. This is perhaps because achieving any reward takes a large amount of exploration and thus episodes that could provide a significant update to the value function are rare. The relationship between the number of messages and the size of the messages remained similar to that demonstrated in Simple Grid.



\begin{figure}[H]
    \centering
    \begin{subfigure}{0.5\textwidth}
        \centering
        \includegraphics[width=\textwidth]{Figures/AlgEval/SimpleGridC_2/episodic_reward.png}
        \caption{The average episodic reward received by agents}
        \label{fig:EvalSGReward}
    \end{subfigure}
    \begin{subfigure}{0.5\textwidth}
        \centering
        \includegraphics[width=\textwidth]{Figures/AlgEval/SimpleGridC_2/num_messages.png}
        \caption{The average number of messages sent from agents to central learner}
        \label{fig:EvalSGNumMessages}
    \end{subfigure}
    \begin{subfigure}{0.5\textwidth}
        \centering
        \includegraphics[width=\textwidth]{Figures/AlgEval/SimpleGridC_2/size_messages.png}
        \caption{The average size of messages sent from agents to central learner}
        \label{fig:EvalSGSizeMessages}
    \end{subfigure}
    \caption{Results from the algorithm evaluation in the Simple Grid environment for DAVIA, DQL, EBCDQL, and ES with $n=5$ over 1000 episodes and 100 trials. The shading represents the $10^\text{th}$ to the $90^{\text{th}}$ percentile across trials.}
    \label{fig:EvalSG}
\end{figure}

\begin{figure}[H]
    \centering
    \begin{subfigure}{0.5\textwidth}
        \centering
        \includegraphics[width=\textwidth]{Figures/AlgEval/FrozenLakeC_2/reward.png}
        \caption{The cumulative average reward received by agents}
        \label{fig:EvalFLReward}
    \end{subfigure}
    \begin{subfigure}{0.5\textwidth}
        \centering
        \includegraphics[width=\textwidth]{Figures/AlgEval/FrozenLakeC_2/num_messages.png}
        \caption{The average number of messages sent from agents to central learner}
        \label{fig:EvalFLNumMessages}
    \end{subfigure}
    \begin{subfigure}{0.5\textwidth}
        \centering
        \includegraphics[width=\textwidth]{Figures/AlgEval/FrozenLakeC_2/size_messages.png}
        \caption{The average size of messages sent from agents to central learner}
        \label{fig:EvalFLSizeMessages}
    \end{subfigure}
    \caption{Results from the algorithm evaluation in the Frozen Lake environment for DAVIA, DQL, EBCDQL, and ES with $n=5$ over 1000 episodes and 100 trials. The shading represents the $10^\text{th}$ to the $90^{\text{th}}$ percentile across trials.}
    \label{fig:EvalFL}
\end{figure}

\section{Limitations}
These experiments were heavily limited by the time it took to run them, this resulted in it being difficult to effectively tune the hyperparameters for each algorithm. Finer tuning could lead to a more accurate comparison of what these algorithms are capable of. In particular EBCDQL and DAVIA could be tuned to communicate more on Frozen Lake which could improve performance.
In addition to this I only tested on two fairly similar environments. On other environments, performance could differ, therefore, using a larger set of environments with more variety between them could give a better picture of algorithm performance. Lastly the experiment consisted of 2 thresholded algorithms but only one black-box algorithm and no gradient compression algorithms. Using a wider variety of algorithms would provide a clearer picture of the relative performance of algorithms within each category and of the performance of each category relative to the others.

\section{Summary}
In summary, we have observed that 
\begin{itemize}
    \item EBCDQL and DAVIA are effective algorithms. They achieve mostly good rewards and their thresholded communication show a good trade off between performance and communication. They send messages only when it provides benefit to the learning process
    \item DQL learns well but has prohibitively poor communication properties. It sends a message every episode which is proportional to the length of the episode
    \item ES is not effective in these environments but has excellent performance in terms of the size of messages sent
\end{itemize}

ES performs poorly in these experiments proving to be brittle in simple environments. However, ES is considerably more flexible than any of the other algorithms tested here. Its effectiveness on environments infamous for their difficulty such as the MuJoCo humanoid shows the performance exhibited here does not well represent its strengths. The size of messages sent are very small, however, the frequency of the messages is poor. One way to improve this is to use a probabilistic communication scheme to reduce the number of messages sent. This is the focus of the subsequent chapter.

% As the simplest algorithm, not designed for communication efficiency, DQL provides a baseline for the other algorithms to compare against. We see that in simple Grid it performes well in terms of quality of learning but proves ineffective on the Frozen lake environment in this regard. Its communication performance is poor with whole trajectories being sent every episode. This is particularly inefficient as the size of these messages grows with episode length. It is also in violation of the prohibition of communication of raw data by FRL. The tabular value function means DQL is limited to use on only simple environments. To improve performance and flexibility one could make use of an approximate value function.

% EBCDQL performs marginally worse than DQL on the Simple Grid environment and better on the Frozen Lake environment. In both cases communication performance is greatly improved due to its adaptive communication scheme without negatively impacting learning performance. However, it is also limited by a tabular value function, scaling poorly with the size of the environment. An adaptation of this method to that with an approximated value function through deep Q learning could prove efficient and capable.

% DAVIA stands out in both learning and communication performance on Simple Grid. It effectively learns a strategy for reaching the terminal and ceases communication after it has done so. On Frozen Lake it did not learn an effective strategy, as it rarely communicated. This could perhaps be remedied by using different basis functions or fine-tuning the hyperparameters. It benefits from the use of an approximated value function meaning it is much more flexible to the size of environments than DQL and EBCDQL. As well as this it enables the use of constant sized messages in the form of value gradients. Extension of this method to non-linear approximations of the value function would create the opportunity apply the communication efficiency of this algorithm to more complex problems.



\bibliographystyle{plain}

\bibliography{Report}

% \appendix

% \chapter{Mathematical proofs}

\section{Uniform distribution of normalized gaussian vectors on the hypersphere}

\begin{theorem}
     Let $\boldsymbol{X} \sim \mathcal{N}(0,I)$
and $\boldsymbol{Z} = \frac{\boldsymbol{X}}{\|\boldsymbol{X}\|_2}$ then $\boldsymbol{Z}$ is uniformly distributed on the unit $n-1$-sphere $S^{n-1}$.

\end{theorem}

\end{document}